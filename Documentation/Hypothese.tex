\documentclass[a4paper,12pt]{scrartcl}

\usepackage[latin1]{inputenc}
\usepackage{ngerman, ae, graphicx, url}
\usepackage{amssymb}
\usepackage{amsmath}
\usepackage{amsfonts}

%opening
\title{Splineapproximation des QRS-Komplex bei fetalen Magnetokardiogrammen}
\author{Hermann Sonntag}

\begin{document}

\maketitle

\begin{abstract}


\end{abstract}

\section{Hypothese}
\subsection{Definitionen}
\begin{itemize}
\item[$H_0$] die Nullhypothese
\item[$H_1$] die alternative Hypothese
\item[$S_{Meas}\left(t\right)$] das gemessenes Signal
\item[$N_{QRS}\left(t\right)$] normiertes Signal des QRS-Komplex
\item[$A$] ein Amplitudenfaktor f"ur die Skalierung des normierten Signals
\item[$n\left(t\right)$] Rauschanteil
\item[$S_v\left(t\right)$] das variable Spitze-Spitze Verh"altnis des Signals
\item[$S_0 + S_1 \cdot t$] linearer Signalanteil
\item[$R\left(t\right)$] das restliche Rauschen, welches f"ur statistische Tests relevant ist.
\end{itemize}
\subsection{Gleichungen}
\begin{eqnarray}
H_0 : S_{Meas}\left(t\right) & = & A \cdot N_{QRS}\left(t\right) + n\left(t\right) \\
H_1: S_{Meas}\left(t\right) & = & A \cdot N_{QRS}\left(t\right) + S_v\left(t\right) + n\left(t\right) \\
n\left(t\right) & = & S_0 + S_1 \cdot t + R\left(t\right)
\end{eqnarray}
\subsection{Methoden}
$
S_{QRS}\left(t\right) = A \cdot N_{QRS}\left(t - t_{beat}\right) + S_0 + S_1 \cdot \left(t - t_{beat}\right) \\
S_{QRS}\left(t\right) = S_{QRS}\left(t, A, S_0, S_1, t_{beat}, N_{QRS}\left(t\right)\right) \\
S_{QRS}\left(t, L\right) = A \cdot N_{QRS}\left(L \cdot t - t_{beat}\right) + S_0 + S_1 \cdot \left(t - t_{beat}\right) \\
$
Um die Positionen der QRS-Komplexe zu detektieren wird die bei dieser Arbeit die Kreuzkorrelation des Originalsignals mit einem gemittelten und normierten QRS-Komplex verwendet. Auch f"ur die Mittelung der QRS-Komplexe eines Kanals wird die Kreuzkorrelation verwendet um die Anfangspositionen der einzelnen Komplexe zu approximieren.
Die allgemeine Kreuzkorrelation ist definiert als:\\
$
\Psi_{xy}\left(\tau\right) = \lim \limits_{T \to \infty} {\frac{1}{2T} \int \limits_{-T}^{T} {x\left(t\right) \cdot y\left(t+\tau\right) \, \mathrm dt}} \\
$
Da bei dieser Arbeit nur diskrete Signale von endlicher L"ange $N$ verarbeitet werden, kann folgende Vereinfachung vorgenommen werden.
$
\text{Wenn } x, y \colon\, \mathbb{Z} \to \mathbb{R} \text{ und } m \in \mathbb{Z} \text{ dann gilt:}\\
\Psi_{xy}\left(m\right) = \lim \limits_{N \to \infty} {\frac{1}{2N + 1}\sum \limits_{i = -N}^{N} { x\left(i\right) \cdot y\left(i + m\right)}} \\
\text{Sei } x(i) := 0 , \forall i \in \mathbb{Z} \setminus [0, N) \text{ und sei} \\
y(i) := 0 , \forall i \in \mathbb{Z} \setminus [0, T_{QRS} \cdot f_s) \text{ und } m \in \mathbb{N}\cup\{0\} \text{ dann}\\
\Psi_{xy}\left(m\right) = \frac{1}{N}\sum \limits_{i = 0}^{N-1} { x\left(i\right) \cdot y\left(i + m\right)} \\
$
Der Zeitvektor, der die Zeitwerte f"ur die Signalwerte enth"alt ist wie folgt definiert:\\
$
\vec{t} = \left ( \begin{array}{c} 0 \\ \Delta t \\ \vdots \\ (N-1) \cdot \Delta t \end{array} \right ) = \left ( \begin{array}{c} t_1 \\ \vdots \\ t_n \end{array} \right )\\
$
Der Vektor der Zeitpunkte der Maxima der Kreuzkorrelation eines normierten und auf die Kanall"ange verl"angerten QRS-Komplexes mit einem Kanal ist:\\
$
\vec{t_{corr}} = \left ( \begin{array}{c} t_{corr_1} \\ \vdots \\ t_{corr_n} \end{array} \right ) \\
\vec{t_0} = \vec{t_{beat}} - \vec{t_{corr}} \\
\vec{m_0} = f_s \cdot \vec{t_0} \\
\vec{m_{beat}} = f_s \cdot \vec{t_{beat}} \\
\vec{m_{corr}} = f_s \cdot \vec{t_{corr}} \\
t_{corr_{0}} := - \frac{1}{2} T_{QRS} \\
\Psi_{xy}\left(t_{corr_{i+1}}\right) := \max_{\tau}\left[\Psi_{xy}\left(\tau\right)\right], \forall \tau \in \left[t_{corr_{i}} + \frac{1}{2} T_{QRS}, t_{corr_{i}} + \frac{3}{2} T_{QRS}\right), \forall i \in \mathbb{N}\cup\{0\} \\
\tilde{\Psi_{xy}}\left(m_{corr_{i+1}}\right) := \max_{m}\left[\sum \limits_{i = 0}^{N-1} { x\left(i\right) \cdot y\left(i + m\right)}\right], \\
\forall m \in \left[m_{corr_{i}} + \frac{1}{2} M_{QRS}, m_{corr_{i}} + \frac{3}{2} M_{QRS}\right), \forall i \in \mathbb{N}\cup\{0\} \\
$
F"ur die bessere Auswertung der Kreuzkorrelation wird diese vom Offset befreit und normiert:\\
$
N_{\vec{f}} = \frac{\vec{f} - \vec{f_{min}}}{f_{max} - f_{min}}\\
N_{QRS}\left(t - t_{corr_i}\right) = \frac{1}{M}\sum \limits_{i=1}^{M} {N_y\left(t\right)}, \forall t \in [t_{corr_i}, t_{corr_i} + T_{QRS}) \\
$
Der normierte und offsetfreie QRS-Komplex wird wie in der Hypothese formuliert mit dem Originalsignal gleichgesetzt. Zur numerischen L"osung dieser Gleichung wird der Levenbergh-Marquard-Algorithmus verwendet, wof"ur die folgende Jacobi-Matrix berechnet werden muss:\\
$
J_{S_{QRS}} = \begin{pmatrix} \frac{\partial S_{QRS_1}}{\partial A} & \frac{\partial S_{QRS_1}}{\partial t_{beat}} & \ldots & \frac{\partial S_{QRS_1}}{\partial x_p} \\ \vdots & \vdots & \ddots & \vdots & \\ \frac{\partial S_{QRS_m}}{\partial A} & \frac{\partial S_{QRS_m}}{\partial t_{beat}} & \ldots & \frac{\partial S_{QRS_m}}{\partial x_p} \end{pmatrix} \\
$
Die Kovarianzmatrix wird verwendet um die G"ute der Sch"atzung zu bestimmen, sie ist wie folgt definiert:\\
$
\Sigma = \begin{bmatrix} \mathrm{E}[(X_1 - \mu_1)(X_1 - \mu_1)] & \mathrm{E}[(X_1 - \mu_1)(X_2 - \mu_2)] & \cdots & \mathrm{E}[(X_1 - \mu_1)(X_n - \mu_n)] \\ \\ \mathrm{E}[(X_2 - \mu_2)(X_1 - \mu_1)] & \mathrm{E}[(X_2 - \mu_2)(X_2 - \mu_2)] & \cdots & \mathrm{E}[(X_2 - \mu_2)(X_n - \mu_n)] \\ \\ \vdots & \vdots & \ddots & \vdots \\ \\ \mathrm{E}[(X_n - \mu_n)(X_1 - \mu_1)] & \mathrm{E}[(X_n - \mu_n)(X_2 - \mu_2)] & \cdots & \mathrm{E}[(X_n - \mu_n)(X_n - \mu_n)] \end{bmatrix}\\ 
$
Als Test f"ur die Richtigkeit der Gleichungsl"osung wird der Chi-quadrat-Test wie folgt verwendet:\\
$
\chi^2 = \sum \limits_{i = 0}^{N-1} { f_i^2 } \\
\frac{\chi^2}{d_f} = \frac{\chi^2}{N-p} 
$
\end{document}
