\documentclass[pdftex,12pt,a4paper]{scrreprt}
\usepackage[onehalfspacing]{setspace}
\usepackage[T1]{fontenc}
\usepackage[utf8]{inputenc}
\usepackage[ngerman]{babel}
\usepackage[pdftex]{graphicx}
\newcommand{\HRule}{\rule{\linewidth}{0.5mm}}



\begin{document}

\begin{titlepage}
 
\begin{center}
 
 
% Upper part of the page
\thispagestyle{empty} ~
{\normalsize\fontfamily{phv}\fontsize{14}{14}\selectfont

\vspace{-2cm}

\begin{figure}[h!]
\includegraphics[width=0.2\hsize]{logo_gr.jpg}\hfill
\begin{minipage}[b]{0.75\hsize}
{\fontseries{b}\fontsize{25pt}{20} \selectfont{%
   Technische Universität Ilmenau}}\\[1ex]
{\fontsize{12pt}{10} \selectfont{%
  Fakultät für Informatik und Automatisierung\\
      Institut für Biomedizinische Technik und Informatik\\
    }}
\end{minipage}
\end{figure}

\vspace{20mm}
}

\large{Studiengang Biomedizinische Technik}
 
% Title
\HRule \\[0.4cm]
{ \huge \bfseries Praktikumsbericht}\\[0.4cm]
 
\HRule \\[1.5cm]
 
% Author and supervisor
\includegraphics[width=0.2\hsize]{header_logo.jpg} GmbH \& Co. KG \\
Henkestraße 91 \\
91052 Erlangen \\
Germany \\
von 15.09.2009 bis 14.02.2010

\vspace{20mm}

\begin{minipage}{0.4\textwidth}
\begin{flushleft} \large
\emph{Author:}\\
Hermann Sonntag \\
41016\\
hermann.sonntag@\\tu-ilmenau.de
\end{flushleft}
\end{minipage}
\begin{minipage}{0.4\textwidth}
\begin{flushright} \large
\emph{Betreuer im Unternehmen:} \\ 
Dipl. Ing. Andreas Bießmann \\
biessmann@corscience.de
\emph{Betreuer im Institut:} \\
Dipl. Ing. Daniel Laqua
\end{flushright}
\end{minipage}
 
\vfill
 
% Bottom of the page
{\large \today}
 
\end{center}
 
\end{titlepage}

\tableofcontents

\begin{abstract}

\end{abstract}

\section{Allgemeines zum Betrieb}
Was macht die Firma/Einrichtung? Wie groß? Branche?\\
Die Firma Corscience entwickelt Elektronik für die Medizintechnik und stellt Medizintechnikprodukte her.
Der Schwerpunkt liegt bei Geräten für die Herzmedizin.
Welche Aufgabe hat "`meine"' Abteilung/Gruppe? Wie groß? Leitungsstruktur?\\
Das Praktikum mache ich in der Entwicklungsabteilung.
Welche Rolle hat mein Betreuer dort?
Mein Betreuer ist der Experte für Betriebssysteme mit dem Schwerpunkt auf embedded Linux in diesem Unternehmen.
Wie/wo/durch wen habe ich die Praktikumsstelle gefunden? Habe ich dort schon vorher gearbeitet?\\
Die Praktikumsstelle fand ich durch Empfehlung eines Kommilitonen.
Erfahrungen dabei oder Lehren daraus?\\
Wie viel Bezahlung habe ich bekommen?\\
Ich bekam 300 Euro pro Monat und 200 Euro für jeden Monat am Ende des Praktikums als Bonus für das Erreichen des Praktikumszieles.
Was für Arbeitszeiten hatte ich?\\
Ich habe eine 40 Stunden Woche, es gibt eine Kernarbeitszeit von 9Uhr bis 16Uhr, in dieser Zeit soll man im Unternehmen sein,
wann man die restliche Zeit arbeitet ist einem weitgehend freigestellt.
\section{Eingliederung}
Mein Arbeitsplatz befindet sich in einem freien Bereich im Unternehmen in der Entwicklungabteilung,
der extra für Praktikanten und Auszubildende eingerichtet wurde.
Ich habe einen von vier Computerarbeitsplätzen, in diesem Bereich arbeiten noch zwei Auszubildende und ein Praktikant/Master-Student.
Neben den Schreibtischen befindet sich ein Lötbereich mit mehreren Lötstationen und ein Arbeitsplatz für andere Elektronikarbeiten.
\section{Aufgabenstellung}
Meine Aufgabenstellung ist es auf einem embedded System eine WLAN-Funktionalität zu integrieren.
Eine wichtige Rolle spielt dabei embedded Linux als Betriebssystem unter dem das WLAN-Modul laufen soll.
Die WLAN-Funktionalität ist später zum Beispiel für die Telemedizin von Bedeutung.
\subsection{Tätigkeitsumfeld}
An welchem Projekt oder welchen Aufgaben habe ich mitgewirkt?\\
Neben meinem Hauptthema hab ich an einem Projekt mitgewirkt, welches sich mit HD-Videoübertragung über WLAN-Standard N beschäftigt.
Da dieses Projekt auch WLAN als Schwerpunkt hat, konnte ich hier gut in die Thematik einsteigen. Wozu sind die da?\\
Das Hauptproblem liegt hier bei den hohen Datenraten, die erreicht werden müssen um das Videosignal ohne große Verzögerung übertrage zu können. Daher mussten viele Messungen des Durchsatzes gemacht werden.
Bei diesem Projekt geht es darum wie HD-Videoaufnahmen von einer Kamera
über WLAN N zu einem Anzeigegerät zu übertragen. Hauptaugenmerkt waren
die besonderen Umweltbedingungen bezüglich EM Abschirmung usw. ...

Wer hat daran noch gearbeitet?\\
Das Projekt wird von einem Projektleiter betreut und es arbeiten einige weitere Mitarbeiter an der Vorbereitung und Umsetzung.
Wie und wie eng habe ich mit denen zusammengearbeitet?\\
Für die Vorbereitung des Projektes mussten Datenratenmessungen durchgeführt werden, dabei arbeitete ich mit einem Mitarbeiter zusammen.
Meine Hauptaufgabe war dabei die Linuxseite der Datenratenmessung aufzubauen.
\subsection{Aufgabe und Ziele}
Was sollte ich aus Sicht der Einrichtung tun?\\
Die erste Aufgabe bei meinem Hauptthema war die Einarbeitung in die Problematik. Ich bekam ein Atmel Board AT91SAM9263-EK welches ich einrichten musste
um darauf einen Linux Kernel über die Netzwerkschnittstelle booten zu können. Um auf dem Board arbeiten zu können musste ich auch ein Filesystem mit den 
notwendigen Programmen erstellen und dieses ebenfalls über den Netzwerkadapter einbinden. Die schwierigste Aufgabe war jetzt den Linux Kernel so zu konfigurieren dass er die Hardware also vorallem die WLAN-Module und gewünschte Methoden richtig unterstützt.
Der erste Schritt für die Umsetzung meiner Aufgabenstellung war die Recherche nach geeigneten WLAN-Modulen auf dem Markt.
Dabei ging es vor allem um folgende Punkte:\\
\begin{itemize}
 \item Gibt es einen Linuxtreiber für die verwendeten Chipsätze der Hardware und befindet der sich im mainline kerneltree?
 \item Welche WLAN-Standards unterstützt die Hardware (IEEE 802.11 a/b/g/n)?
 \item Welche Schnittstellen werden unterstützt/benötigt (SDIO/SPI oder PCI ...)
\end{itemize}

Was sollte ich im Verlauf des Praktikums erreichen (fremde (Arbeits)Ziele)?\\
Das Ziel war es eine Hardware zu finden welche über eine verfügbare Schnittstelle des Atmel Boards AT91SAM9263-EK (bevorzugt SDIO oder SPI) mit diesem kommuniziert, dieses anzuschließen und unter embedded Linux verfügbar zu machen.

Was wollte ich im Verlauf des Praktikums erreichen (eigene (Lern)Ziele)?\\
Meine Ziele für dieses Praktikum waren:
\begin{itemize}
 \item Einblicke in den Linux-Kernel zu bekommen, das heißt Einblicke in den Aufbau in Konfigurationsmöglichkeiten und Anpassbarkeit des Kernels.
 \item Erlernen von Linux-Treiberentwicklung mit dem Schwerpunkt auf Netzwerktreiber.

\end{itemize}

\section{Problemlösung}
Bei der Recherche zu verfügbaren WLAN-Modulen entstand folgende Tabelle.
\begin{center}
\includegraphics[angle=90, width=4cm,height=25cm]{tabelle-img1.png}
\end{center}

\subsection{Tätigkeiten und Arbeitsergebnisse}
[[Dies hier sollte meist der längste und ausführlichste Abschnitt sein. Fügen Sie ggf.
Unterabschnitte ein.]]\\
Was habe ich konkret getan?\\
Welche Schwierigkeiten habe ich dabei überwunden?\\
Welche nicht?\\

\section{Endergebnis}
Was ist insgesamt herausgekommen?
Wie vergleicht sich das mit den Zielen? Ist es insgesamt ein Erfolgserlebnis oder nicht?\\
Wenn nicht, woran hat es gehapert?\\
\section{Resüme}
\section{Einsichten und Fazit}
Dieser Abschnitt beschreibt, welche wichtigsten Einsichten (Aha-Erlebnisse) ich aus dem
Praktikum mitgenommen habe.

\subsection{Technick}
besondere Eigenschaften, Stärken, Schwächen, Fallen etc. der Technologien, mit denen
ich gearbeitet habe: Sprachen, Bibliotheken, Plattformen, Werkzeuge, Anwendungen etc.
(evtl. auch außerhalb der Softwarewelt)\\
Dieser Abschnitt kann in seltenen Fällen leer sein.

\subsection{Methodik}
[Einsichten über Arbeits- und Vorgehensweisen: Wenn man im Kontext X versucht Y zu
tun, dann sollte man unbedingt A beachten/tun/vermeiden bzw. möglichst versuchen wie
B vorzugehen/nicht vorzugehen, weil C. Mein Erlebnis in diesem Zusammenhang war D.
(Je mehr solcher Einsichten Sie hatten, desto besser war das Praktikum.)\\
Dieser Abschnitt kann eventuell auch leer sein.

\subsection{Sonstiges}
Einsichten darüber, wie Firmen/Gruppen/Projekte funktionieren oder nicht funktionieren;
wie Menschen agieren oder nicht agieren, wenn sie zusammenarbeiten oder
zusammenarbeiten sollten; wo ich mich selbst falsch eingeschätzt habe; wo ich mich
selbst unter- oder überschätzt habe; etc. pp.]\\
Dieser Abschnitt kann eventuell auch leer sein.

\subsection{Fazit}
Was hat mir das Praktikum gebracht, das ich im Studium nicht oder weniger bekomme?\\
Was bringt mir das Studium, das ich im Praktikum nicht oder kaum bekommen kann?\\
Was will ich deshalb künftig an meinem Studierverhalten verändern?\\
Welche Tipps gebe ich anderen, die ein Praktikum suchen, worauf sie achten sollten?\\

\end{document}          
