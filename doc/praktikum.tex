\documentclass[pdftex,12pt,a4paper]{scrreprt}
\usepackage[onehalfspacing]{setspace}
\usepackage[T1]{fontenc}
\usepackage[utf8]{inputenc}
\usepackage[ngerman]{babel}
\usepackage[pdftex]{graphicx}
\newcommand{\HRule}{\rule{\linewidth}{0.5mm}}



\begin{document}

\begin{titlepage}
 
\begin{center}
 
 
% Upper part of the page
\thispagestyle{empty} ~
{\normalsize\fontfamily{phv}\fontsize{14}{14}\selectfont

\vspace{-2cm}

\begin{figure}[h!]
\includegraphics[width=0.2\hsize]{logo_gr.jpg}\hfill
\begin{minipage}[b]{0.75\hsize}
{\fontseries{b}\fontsize{25pt}{20} \selectfont{%
   Technische Universität Ilmenau}}\\[1ex]
{\fontsize{12pt}{10} \selectfont{%
  Fakultät für Informatik und Automatisierung\\
      Institut für Biomedizinische Technik und Informatik\\
    }}
\end{minipage}
\end{figure}

\vspace{20mm}
}

\large{Studiengang Biomedizinische Technik}
 
% Title
\HRule \\[0.4cm]
{ \huge \bfseries Praktikumsbericht}\\[0.4cm]
 
\HRule \\[1.5cm]
 
% Author and supervisor
\includegraphics[width=0.2\hsize]{header_logo.jpg} GmbH \& Co. KG \\
Henkestraße 91 \\
91052 Erlangen \\
Germany \\
von 15.09.2009 bis 14.02.2010

\vspace{20mm}

\begin{minipage}{0.4\textwidth}
\begin{flushleft} \large
\emph{Author:}\\
Hermann Sonntag \\
41016\\
hermann.sonntag@\\tu-ilmenau.de
\end{flushleft}
\end{minipage}
\begin{minipage}{0.4\textwidth}
\begin{flushright} \large
\emph{Betreuer im Unternehmen:} \\ 
Dipl. Ing. Andreas Bießmann \\
biessmann@corscience.de
\emph{Betreuer im Institut:} \\
Dipl. Ing. Daniel Laqua
\end{flushright}
\end{minipage}
 
\vfill
 
% Bottom of the page
{\large \today}
 
\end{center}
 
\end{titlepage}

\tableofcontents

\begin{abstract}

\end{abstract}

\section{Allgemeines zum Betrieb}
Was macht die Firma/Einrichtung? Wie groß? Branche?\\
Die Firma entwickelt Elektronik für die Medizintechnik und stellt Medizintechnikprodukte her.
Der Schwerpunkt liegt bei Geräten für die Herzmedizin.
Welche Aufgabe hat "`meine"' Abteilung/Gruppe? Wie groß? Leitungsstruktur?\\
Welche Rolle hat mein Betreuer dort?
Mein Betreuer ist der Experte für Linux und embedded Linux in diesem Unternehmen.
Wie/wo/durch wen habe ich die Praktikumsstelle gefunden? Habe ich dort schon vorher gearbeitet?\\
Die Praktikumsstelle fand ich durch Empfehlung eines Kommilitonen.
Erfahrungen dabei oder Lehren daraus?\\
Wie viel Bezahlung habe ich bekommen?\\
Ich bekam 300 Euro pro Monat und 200 Euro für jeden Monat am Ende des Praktikums als Bonus für das Erreichen des Praktikumszieles.
Was für Arbeitszeiten hatte ich?\\
Ich habe eine 40 Stunden Woche, es gibt eine Kernarbeitszeit von 9Uhr bis 16Uhr, in dieser Zeit soll man im Unternehmen sein,
wann man die restliche Zeit arbeitet ist einem freigestellt.
\section{Eingliederung}
\section{Aufgabenstellung}
\subsection{Tätigkeitsumfeld}
An welchem Projekt oder welchen Aufgaben habe ich mitgewirkt?\\
Wozu sind die da?\\
Wer hat daran noch gearbeitet?\\
Wie und wie eng habe ich mit denen zusammengearbeitet?\\

\subsection{Aufgabe und Ziele}
Was sollte ich aus Sicht der Einrichtung tun?\\
Was sollte ich im Verlauf des Praktikums erreichen (fremde (Arbeits)Ziele)?\\
Was wollte ich im Verlauf des Praktikums erreichen (eigene (Lern)Ziele)?\\

\section{Problemlösung}
\subsection{Tätigkeiten und Arbeitsergebnisse}
[[Dies hier sollte meist der längste und ausführlichste Abschnitt sein. Fügen Sie ggf.
Unterabschnitte ein.]]\\
Was habe ich konkret getan?\\
Welche Schwierigkeiten habe ich dabei überwunden?\\
Welche nicht?\\

\section{Endergebnis}
Was ist insgesamt herausgekommen?
Wie vergleicht sich das mit den Zielen? Ist es insgesamt ein Erfolgserlebnis oder nicht?\\
Wenn nicht, woran hat es gehapert?\\
\section{Resüme}
\section{Einsichten und Fazit}
Dieser Abschnitt beschreibt, welche wichtigsten Einsichten (Aha-Erlebnisse) ich aus dem
Praktikum mitgenommen habe.

\subsection{Technick}
besondere Eigenschaften, Stärken, Schwächen, Fallen etc. der Technologien, mit denen
ich gearbeitet habe: Sprachen, Bibliotheken, Plattformen, Werkzeuge, Anwendungen etc.
(evtl. auch außerhalb der Softwarewelt)\\
Dieser Abschnitt kann in seltenen Fällen leer sein.

\subsection{Methodik}
[Einsichten über Arbeits- und Vorgehensweisen: Wenn man im Kontext X versucht Y zu
tun, dann sollte man unbedingt A beachten/tun/vermeiden bzw. möglichst versuchen wie
B vorzugehen/nicht vorzugehen, weil C. Mein Erlebnis in diesem Zusammenhang war D.
(Je mehr solcher Einsichten Sie hatten, desto besser war das Praktikum.)\\
Dieser Abschnitt kann eventuell auch leer sein.

\subsection{Sonstiges}
Einsichten darüber, wie Firmen/Gruppen/Projekte funktionieren oder nicht funktionieren;
wie Menschen agieren oder nicht agieren, wenn sie zusammenarbeiten oder
zusammenarbeiten sollten; wo ich mich selbst falsch eingeschätzt habe; wo ich mich
selbst unter- oder überschätzt habe; etc. pp.]\\
Dieser Abschnitt kann eventuell auch leer sein.

\subsection{Fazit}
Was hat mir das Praktikum gebracht, das ich im Studium nicht oder weniger bekomme?\\
Was bringt mir das Studium, das ich im Praktikum nicht oder kaum bekommen kann?\\
Was will ich deshalb künftig an meinem Studierverhalten verändern?\\
Welche Tipps gebe ich anderen, die ein Praktikum suchen, worauf sie achten sollten?\\

\end{document}          
