\documentclass[pdftex,12pt,a4paper]{scrartcl}
\usepackage[onehalfspacing]{setspace}
\usepackage[T1]{fontenc}
\usepackage[utf8]{inputenc}
\usepackage[ngerman]{babel}
\usepackage[pdftex]{graphicx}
\usepackage{graphicx}
\usepackage{eso-pic}
\usepackage[pdftex]{pdfpages}
\usepackage{float}
\newcommand{\HRule}{\rule{\linewidth}{0.5mm}}

\begin{document}

\begin{titlepage}
 
\begin{center}
 
 
% Upper part of the page
\thispagestyle{empty} ~
{\normalsize\fontfamily{phv}\fontsize{14}{14}\selectfont

\vspace{-2cm}

\begin{figure}[h!]
\includegraphics[width=0.2\hsize]{logo_gr.jpg}\hfill
\begin{minipage}[b]{0.75\hsize}
{\fontseries{b}\fontsize{25pt}{20} \selectfont{%
   Technische Universität Ilmenau}}\\[1ex]
{\fontsize{12pt}{10} \selectfont{%
  Fakultät für Informatik und Automatisierung\\
      Institut für Biomedizinische Technik und Informatik\\
    }}
\end{minipage}
\end{figure}

\vspace{20mm}
}

\large{Studiengang Biomedizinische Technik}
 
% Title
\HRule \\[0.4cm]
{ \huge \bfseries Praktikumsbericht}\\[0.4cm]
 
\HRule \\[1.5cm]
 
% Author and supervisor
\includegraphics[width=0.2\hsize]{header_logo.jpg} GmbH \& Co. KG \\
Henkestraße 91 \\
91052 Erlangen \\
Germany \\
von 15.09.2009 bis 14.02.2010

\vspace{20mm}

\begin{minipage}{0.4\textwidth}
\begin{flushleft} \large
\emph{Author:}\\
Hermann Sonntag \\
41016\\
hermann.sonntag@\\tu-ilmenau.de
\end{flushleft}
\end{minipage}
\begin{minipage}{0.4\textwidth}
\begin{flushright} \large
\emph{Betreuer im Unternehmen:} \\ 
Dipl. Ing. Andreas Bießmann \\
biessmann@corscience.de
\emph{Betreuer im Institut:} \\
Dipl. Ing. Daniel Laqua
\end{flushright}
\end{minipage}
 
\vfill
 
% Bottom of the page
{\large \today}
 
\end{center}
 
\end{titlepage}

\tableofcontents
%\begin{abstract}
%\end{abstract}
\section{Allgemeines zum Betrieb}
%Was macht die Firma/Einrichtung? Wie groß? Branche?\\
Die Firma Corscience entwickelt Elektronik für die Medizintechnik und stellt Medizintechnikprodukte her.
Der Schwerpunkt liegt bei Geräten für die Herzmedizin.
%Welche Aufgabe hat "`meine"' Abteilung/Gruppe? Wie groß? Leitungsstruktur?\\
Das Praktikum absolvierte ich in der Entwicklungsabteilung, hier geht es um die Entwicklung neuer Technologien für verschiedene Projekte und um die Verbesserung bestehender Konzepte und Produkte.
Die Praktikumsstelle fand ich durch Empfehlung eines Kommilitonen, ich kannte die Firma davor noch nicht.

Ein Praktikum wird bei Corscience wird mit 300 Euro pro Monat vergütet. Des Weiteren wird in der Regel eine Erfolgsprämie von 200 Euro pro geleistetem Monat vereinbart, welche zum Ende des Praktikums ausgezahlt wird.
%Was für Arbeitszeiten hatte ich?\\
Es gibt bei Corscience eine Kernarbeitszeit von 9Uhr bis 16Uhr, in dieser Zeit soll man im Unternehmen sein, am Freitag geht die Kernarbeitszeit nur bis 15Uhr, wann man die restliche Zeit arbeitet ist einem weitgehend freigestellt, solange die gesamte Arbeitszeit pro Woche 40 Stunden beträgt.

Mein Praktikumsbetreuer ist in diesem Unternehmen der Experte für Betriebssysteme mit dem Schwerpunkt auf embedded Linux.

%Erfahrungen dabei oder Lehren daraus?\\
Während des Praktikums habe ich viel über den Umgang mit und Aufbau von Linux kennengelernt, des weiteren habe ich vor allem Erfahrungen im Bereich der C++ Programmierung erworben und viele neue Tools kennengelernt. Die Arbeit mit dem Atmel-Board gab mir Einblicke in die Treiberentwicklung für den Linux-Kernel und somit in die Kernelspaceprogrammierung. Eine der wichtigsten Erfahrungen war auch das Inbetriebnehmen des Atmel-Boards mit einem selbst konfigurierten Betriebssystem (mit Linux-Kernel). Da das Hauptthema im Bereich WLAN lag habe ich dort auch sehr viel, für mich neues, dazugelernt. Beim Anschließen der Hardware habe ich Löterfahrungen gesammelt da der Stecker des Moduls sehr klein war und viele PINs hatte.
\section{Eingliederung}
Mein Arbeitsplatz befindet sich in einem freien Bereich im Unternehmen in der Entwicklungabteilung,
der extra für Praktikanten und Auszubildende eingerichtet wurde. Dort hatte ich einen von vier Computerarbeitsplätzen, an denen außer mir noch zwei Auszubildende und ein Praktikant/Master-Student arbeiteten.
Neben den Schreibtischen befindet sich ein Lötbereich mit mehreren Lötstationen und ein Arbeitsplatz für andere Elektronikarbeiten in diesem Bereich.
\section{Aufgabenstellung}
Meine Aufgabenstellung war es auf einem embedded System eine WLAN-Funktionalität zu integrieren.
Eine wichtige Rolle spielte dabei embedded Linux als Betriebssystem unter dem das WLAN-Modul laufen soll.
Die WLAN-Funktionalität ist später zum Beispiel für die Telemedizin von Bedeutung.
\subsection{Tätigkeitsumfeld}
%An welchem Projekt oder welchen Aufgaben habe ich mitgewirkt?\\
Neben meinem Hauptthema habe ich an einem Projekt mitgewirkt, welches sich mit HD-Videoübertragung über WLAN-Standard N beschäftigt.
Da dieses Projekt auch WLAN als Schwerpunkt hat, konnte ich hier gut in die Thematik einsteigen.
%Wozu sind die da?\\
Das Hauptproblem liegt hier bei den hohen Datenraten, die erreicht werden müssen um das Videosignal ohne große Verzögerung übertrage zu können. Daher mussten viele Messungen des Durchsatzes gemacht werden.
Bei diesem Projekt geht es darum HD-Videoaufnahmen von einer Kamera
über WLAN N zu einem Anzeigegerät zu übertragen. Hauptaugenmerkt waren
die besonderen Umweltbedingungen bezüglich EM Abschirmung usw. Deshalb wurden verschiedene Messszenarien untersucht.

%Wer hat daran noch gearbeitet?\\
Das Projekt wird von einem Projektleiter betreut und es arbeiten einige weitere Mitarbeiter an der Vorbereitung und Umsetzung.

%Wie und wie eng habe ich mit denen zusammengearbeitet?\\
Für die Vorbereitung des Projektes mussten Datenratenmessungen durchgeführt werden, dabei arbeitete ich mit einem Mitarbeiter zusammen.
Meine Hauptaufgabe war dabei die Linuxseite der Datenratenmessung aufzubauen.

Des weiteren schrieb ich während des Praktikums ein Programm zur Visualisierung eines 12-Kanal-EKGs in C++, für das ich im Laufe der Zeit weitere Funktionen hinzufügte und am Ende der Praktikumszeit auch eine Anpassung an die Windows-Plattform.
Das nächste Bild zeigt den EKG-Verstärker, der die Daten der Kanäle über die Serielle Schnittstelle überträgt.
\begin{figure}[H] %  figure placement: here, top, bottom, or page
    \centering
    \includegraphics[angle=90, width=480px,height=690px]{ecg.jpg}
    \caption{verwendeter EKG-Verstärker mit serieller Schnittstelle}
\end{figure}
\subsection{Aufgabe und Ziele}
%Was sollte ich aus Sicht der Einrichtung tun?\\
Die erste Aufgabe bei meinem Hauptthema war die Einarbeitung in die Problematik.
Ich bekam ein Atmel Board AT91SAM9263-EK welches ich einrichten musste um darauf einen Linux Kernel über die Netzwerkschnittstelle booten zu können.
Um auf dem Board arbeiten zu können musste ich auch ein Filesystem mit den notwendigen Programmen erstellen und dieses ebenfalls über den Netzwerkadapter einbinden.
\begin{figure}[H] %  figure placement: here, top, bottom, or page
    \centering
    \includegraphics[angle=90, width=480px,height=690px]{board3.jpg}
    \caption{Atmel-Board ohne WLAN-Modul}
\end{figure}
Die schwierigste Aufgabe war jetzt den Linux Kernel so zu konfigurieren dass er die Hardware also vorallem die WLAN-Module und gewünschte Methoden richtig unterstützt.
\begin{figure}[H] %  figure placement: here, top, bottom, or page
    \centering
    \includegraphics[angle=90, width=480px,height=690px]{alle.jpg}
    \caption{verwendete WLAN-Karten}
\end{figure}
\begin{figure}[H] %  figure placement: here, top, bottom, or page
    \centering
    \includegraphics[angle=90, width=480px,height=690px]{atheros.jpg}
    \caption{Atheros-Stick}
\end{figure}
\begin{figure}[H] %  figure placement: here, top, bottom, or page
    \centering
    \includegraphics[angle=90, width=480px,height=690px]{ralink.jpg}
    \caption{Ralink-Stick}
\end{figure}
\begin{figure}[H] %  figure placement: here, top, bottom, or page
    \centering
    \includegraphics[angle=0, height=240px,width=345px]{router.jpg}
    \caption{Router}
\end{figure}
\begin{figure}[H] %  figure placement: here, top, bottom, or page
    \centering
    \includegraphics[angle=0, height=240px,width=345px]{router_back.jpg}
    \caption{Router-Rückseite mit Credentials}
\end{figure}
Der erste Schritt für die Umsetzung meiner Aufgabenstellung war die Recherche nach geeigneten WLAN-Modulen auf dem Markt.
Dabei ging es vor allem um folgende Punkte:\\
\begin{itemize}
 \item Gibt es einen Linuxtreiber für die verwendeten Chipsätze der Hardware und befindet der sich im mainline kerneltree?
 \item Welche WLAN-Standards unterstützt die Hardware (IEEE 802.11 a/b/g/n)?
 \item Welche Frequenzbänder werden unterstützt (2.4/5GHz)
 \item Welche Schnittstellen werden unterstützt/benötigt (SDIO/SPI oder PCI ...)
 \item Preis
 \item Zertifikate
 \item Verfügbarkeit
 \item Langzeitverfügbarkeit
 \item Wie wird das Modul angebracht (Handhabung)
\end{itemize}

%Was sollte ich im Verlauf des Praktikums erreichen (fremde (Arbeits)Ziele)?\\
Das Ziel war es eine Hardware zu finden welche über eine verfügbare Schnittstelle des Atmel Boards AT91SAM9263-EK (bevorzugt SDIO oder SPI) mit diesem kommuniziert, dieses anzuschließen und unter embedded Linux verfügbar zu machen.

%Was wollte ich im Verlauf des Praktikums erreichen (eigene (Lern)Ziele)?\\
Meine Ziele für dieses Praktikum waren:
\begin{itemize}
 \item Einblicke in den Linux-Kernel zu bekommen, das heißt Einblicke in den Aufbau in Konfigurationsmöglichkeiten und Anpassbarkeit des Kernels.
 \item Erlernen von Linux-Treiberentwicklung mit dem Schwerpunkt auf Netzwerktreiber.
\end{itemize}
%\begin{center}
%\includegraphics[angle=90, width=480px,height=690px]{oxy.jpg}
%\end{center}
\section{Problemlösung}
Bei der Recherche zu verfügbaren WLAN-Modulen entstand folgende Tabelle.
\begin{center}
    \includegraphics[angle=90, height=25cm]{tabelle.png}
\end{center}
\subsection{Tätigkeiten und Arbeitsergebnisse}
%[[Dies hier sollte meist der längste und ausführlichste Abschnitt sein. Fügen Sie ggf.
%Unterabschnitte ein.]]\\
%Was habe ich konkret getan?\\
Für die Recherche habe ich zuerst auf der Linux-Wireless Homepage Informationen über vorhandene Treiber, Geräte und Anbieter gesucht. Auf dieser Seite gab es sehr viele Informationen zu den Treibern aber wenig zur eigentlichen Hardware, problematisch war auch dass es hier keine direkten Ergebnisse für die geforderte SDIO/SPI-Schnittstelle gab. Über Chipsätze und Links zu anderen Seiten kam ich dann auch zu Modulen mit der geforderten Schnittstelle. Auch die Seiten der einzelnen Chip- bzw. Modulhersteller wie Ralink, Atheros, Marvell und Ähnliche hatten Informationen über Module für die gewünschte Schnittstelle. Das Problem bestand nun darin herauszufinden ob Treiber für Linux vorhanden sind, weiterentwickelt werden, unter einer geeigneten open source Lizenz stehen, für eine aktuelle Kernelversion zur Verfügung stehen und bestenfalls im Mainline-Kernel integriert sind. Es entstand nun eine Tabelle mit verschiedenen WLAN-Modulen die die Anforderungen mehr oder weniger erfüllten. Die Bewertung dieser Module nach der Bewertungstabelle ergab nun, dass das WiBear-Modul das geeigneteste Modul ist.
Es gibt nur wenige Treiber im Mainline-Kernel, die WLAN über die SDIO-Schnittstelle unterstützen und der libertas-Treiber für Marvell Chipsätze tut dies zum Beispiel über das Kernelmodul libertas\_sdio. Das war ein wichtiger Grund für mich, dass WiBear-Modul mit Marvell Chipsatz zu bevorzugen. Weitere Gründe gehen aus der Bewertungstabelle hervor. Zusätzlich unterstützt das WiBear-Modul Bluetooth, was für die Bewertung keine Rolle spielte jedoch ein willkommener Zusatz war. Da das ausgewählte Modul von seinen Zertifizierungen und seiner Spezifikation auch für die Medizintechnik geeignet ist bin ich mit der Auswahl sehr zufrieden.
\begin{figure}[H] %  figure placement: here, top, bottom, or page
    \centering
    \includegraphics[angle=90, width=480px,height=690px]{adapter1.jpg}
    \caption{Adapter von SDIO zu Stecker für das WLAN-Modul}
\end{figure}
\begin{figure}[H] %  figure placement: here, top, bottom, or page
    \centering
    \includegraphics[angle=90, width=480px,height=690px]{adapter2.jpg}
    \caption{Adapter mit Dummy-SD-Karte (damit die Kartenschnittstelle aktiviert wird)}
\end{figure}
%Welche Schwierigkeiten habe ich dabei überwunden?\\
%Welche nicht?\\
\section{Endergebnis}
%Was ist insgesamt herausgekommen?\\
Das ausgewählte WLAN-Modul wurde über die SDIO-Schnittstelle mit einem selbstgelöteten Adapter, wie oben zu sehen, an das Atmel-Board angeschlossen und mit einem Linux (mainline kernel) in Betrieb genommen.\\
\begin{figure}[H] %  figure placement: here, top, bottom, or page
    \centering
    \includegraphics[angle=90, width=480px,height=690px]{board2.jpg}
    \caption{Atmel-Board mit WLAN-Modul}
\end{figure}
\begin{figure}[H] %  figure placement: here, top, bottom, or page
    \centering
    \includegraphics[angle=90, width=480px,height=690px]{adapter3.jpg}
    \caption{Adapteranbringung}
\end{figure}

%Wie vergleicht sich das mit den Zielen? Ist es insgesamt ein Erfolgserlebnis oder nicht?\\
Insgesamt wurde damit das Praktikumsziel erreicht, leider fehlte die Zeit für weitere Benchmarks.\\
%Wenn nicht, woran hat es gehapert?\\
\section{Resüme}
Da das ausgewählte Modul alle geforderten Spezifikationen erfüllte und auf dem Board, welches ich zur Verfügung hatte, funktionierte, sehe ich das Praktikumsergebnis als Erfolg. Ich habe mit dem Modul viele Tests mit verschiedenen Treibern und Treiberbibliotheken gemacht, dabei verschiedene Kernels verwendet und gepatched (Treiber gepatched), und dadurch Sicherheit im Umgang mit der Thematik gewonnen.
\section{Einsichten und Fazit}
%Dieser Abschnitt beschreibt, welche wichtigsten Einsichten (Aha-Erlebnisse) ich aus dem
%Praktikum mitgenommen habe.
\subsection{Technick}
%besondere Eigenschaften, Stärken, Schwächen, Fallen etc. der Technologien, mit denen
%ich gearbeitet habe: Sprachen, Bibliotheken, Plattformen, Werkzeuge, Anwendungen etc.
%(evtl. auch außerhalb der Softwarewelt)\\
%Dieser Abschnitt kann in seltenen Fällen leer sein.
Für die Erstellung des Root-Filesystems für das Betriebssystem auf dem Board verwendete ich buildroot. Jeden Quellcode den ich für das Board kompilierte einschließlich der Kernel, kompilierte ich auch mit den Kompiler von buildroot. Zur Quellcodeverwaltung/Revisionsverwaltung verwendete ich svn oder git. Ich habe in den Sprachen C/C++ entwickelt und verschiedene Bibliotheken wie libnl, fltk und das control-interface von hostap verwendet. Am Ende des Praktikums habe ich mich intensiv mit WPS (WIFI Protected Setup) beschäftigt und mithilfe von hostap eine eigene Methode entwickelt um die zur Verbindung erforderlichen Daten über ein serielles Kabel im WPS-Format zu übertragen.
\begin{figure}[H] %  figure placement: here, top, bottom, or page
    \centering
%\includegraphics[angle=0,width=394px,height=478px]{Konzeptvorstellung1_wps_wire.jpg}
\includegraphics[width=12cm]{Konzeptvorstellung1_wps_wire.jpg}
    \caption{Ausschnitt des Konzeptes für die Übertragung von WPS-Daten über ein serielles Kabel}
\end{figure}
\subsection{Methodik}
%[Einsichten über Arbeits- und Vorgehensweisen: Wenn man im Kontext X versucht Y zu
%tun, dann sollte man unbedingt A beachten/tun/vermeiden bzw. möglichst versuchen wie
%B vorzugehen/nicht vorzugehen, weil C. Mein Erlebnis in diesem Zusammenhang war D.
%(Je mehr solcher Einsichten Sie hatten, desto besser war das Praktikum.)\\
%Dieser Abschnitt kann eventuell auch leer sein.
Zur Vorgehensweise bei der Erarbeitung oder Einarbeitung in Techniken, die einem neu sind, habe ich gelernt weniger einfach zu probieren oder zu kopieren, wie ich es vorher oft gemacht habe und statt dessen Dokumentationen und Quellcode gründlich zu analysieren um ans Ziel zu gelangen.
Bei eigenen Projekten ist es daher auch wichtig viel Dokumentation zu erstellen um sich selbst besser im eigenen Projekt zurecht zu finden und um anderen den einfachen Zugang zu ermöglichen.
\subsection{Sonstiges}
%Einsichten darüber, wie Firmen/Gruppen/Projekte funktionieren oder nicht funktionieren;
%wie Menschen agieren oder nicht agieren, wenn sie zusammenarbeiten oder
%zusammenarbeiten sollten; wo ich mich selbst falsch eingeschätzt habe; wo ich mich
%selbst unter- oder überschätzt habe; etc. pp.]\\
%Dieser Abschnitt kann eventuell auch leer sein.
Bei der Organisation meiner Arbeiten ist mir aufgefallen, dass es mir schwer fällt den Zeitaufwand für eine Aufgabe richtig einzuschätzen und dass es wichtig ist ein gutes Zeitmanagement zu haben um Projekte gut abzuschließen.
\subsection{Fazit}
%Was hat mir das Praktikum gebracht, das ich im Studium nicht oder weniger bekomme?\\
Im Praktikum konnte ich sehr tief in eine Thematik einsteigen, was so im Studium bisher nicht möglich war. Außerdem habe ich vieles selbstständig erledigt und konnte eigene Entscheidungen zu meinem Thema treffen, dies fehlt im Studium.

%Was bringt mir das Studium, das ich im Praktikum nicht oder kaum bekommen kann?\\
Das Studium gibt einem im Gegensatz zum Praktikum die Sicherheit, dass es eine Lösung für das Problem gibt, die einem auch gezeigt wird wenn man sie nicht selber findet.

%Was will ich deshalb künftig an meinem Studierverhalten verändern?\\
In Zukunft werde ich Versuchen die Fragen/Aufgaben im Studium mehr auf eigenem Weg und mit selbst erarbeiteten oder gefundenen Werkzeugen zu lösen.

%Welche Tipps gebe ich anderen, die ein Praktikum suchen, worauf sie achten sollten?\\
Ich denke es ist wichtig dass das Praktikumsthema zu einem passt und dass man interessiert an der Lösung ist um Spaß und Erfolg im Praktikum zu haben.

\appendix
\includepdf[pages={1,2}]{WiBearenglv08.pdf}

\end{document}
